\documentclass[seminar]{fer}

\usepackage[authoryear]{natbib}
\title{Detekcija pješaka u urbanim okruženjima korišenjem značajki temeljenih na teksturi i boji}
\author{Iva Miholić, Gustav Matula, Kristijan Franković, Tomislav Kiš}

\begin{document}
\maketitle

\chapter{Detekcija pješaka u urbanim okruženjima}
\section{Opis zadatka}
Detekcija pješaka u urbanim okruženjima problem je povezan sa automobilskom industrijom, sigurnošću u prometu, sigurnosnim nadziranjem i robotikom. Rješava se kao detekcija objekta u okviru područja računalnog vida. Ovaj projektni zadatak obuhvaća izgradnju detektora pješaka na fotografijama iz urbanih okruženja korištenjem značajki temeljenih na teksturi i boji.

Pregled najznačajnijih rješenja ovog problema dan je u \cite{BenensonOHS14}. Prvi značajniji napredak u detekciji pješaka bila je primjena \emph{VJ} detektora objekata \cite{VJ} na ovaj problem. Detektori temeljeni na histogramu usmjerenih gradijenata \engl{Histogram of Oriented Gradients, HOG} \cite{HOG}  uz linearni ili nelinearni skup potpornih vektora, postigli su značajne rezultate, posebice u kombinaciji sa drugim značajkama temeljenih na svojstvima boje, tekstura i oblika. Više o HOG pristupu bit će riječi u sljedećim poglavljima jer ćemo ga koristiti kao kostur naše implementacije.

Od ostalih rezultata, potrebno je izdvojiti postupke temeljene na modelu rastavljivih dijelova  \engl{Deformable Part Models, DPM} \ref{DPM} u kojem se detekcija dijelova tijela sažima u detekciju cijelog pješaka te nelinearne postupke učenja temeljenih na neuronskim mrežama i stabalima odluke \cite{BenensonOHS14}. Takvi složeniji postupci uspoređeni su sa linearnim SVM-om uz HOG i druge značajke i nisu dali značajno bolji rezultat upućujući na korektnost naše odluke o arhitekturi detektora.

Pristup na koji ćemo se fokusirati koristi metodu skalabilnog kliznog prozora. Unutar prozora na fotografiji, određuju se značajke te se skup piksela unutar prozora binarno klasificira kao fotografija pješaka. Prozor zatim "putuje" po fotografiji testirajući druge prozore te se može dodatno skalirati nakon čega se ponavlja isti postupak. Primjer klasificirane fotografije ovakvim postupkom vidljiv je na slici \ref{primjer_klasifikacije}. Zeleni okviri prikazuju okvire prozora koji su klasificirani kao prikaz pješaka. 

\chapter{Baza podataka za treniranje i verifikaciju rješenja}

Za treniranje i verifikaciju rješenja koristit ćemo skup podataka INRIA \cite{DT05}. To je skup fotografija urbanog okruženja u boji i pripadnih anotacija. Za svaku fotografiju zabilježen je skup graničnih prozora \engl{bounding window} unutar kojih se nalaze prikazi uspravnih osoba. %TODO primjer oznacene slike i kako se definira bounding window

INRIA se do sada često koristio za trening detektora zbog raznolikosti pozadinskih okruženja osoba na slikama i točnosti anotacija \cite{BenensonOHS14}. Za razliku od ostalih javno dostupnih baza podataka za detekciju pješaka, ove fotografije nisu dobivene iz videa te su relativno visoke kvalitete i fotografirane su iz različitih točaka gledišta. U drugim bazama podataka, pješaci su većinom konecentrirani u centralnoj horizontali fotografije jer je ista dobivena iz vozačeve perspektive.

Baza se sastoji od podskupa za trening i podskupa za evaluaciju. U podskupu za trening označena su $1208$ pješaka u $614$ od $1832$ fotografije. U podskupu za testiranje označeno je $566$ pješaka u $453$ od $741$ fotografije \cite{Dollar:2012:PDE:2197081.2197275}

\chapter{Idejno rješenje i prikaz arhitekture sustava računalnog vida}

\section{Pregled značajki temeljenih na teksturi i boji}

\section{Plan arhitekture sustava računalnog vida}
\bibliographystyle{plain}
\bibliography{bibliografija}
\end{document}