\documentclass[seminar]{fer}

\usepackage[authoryear]{natbib}
\title{Detekcija pješaka u urbanim okruženjima korišenjem značajki temeljenih na teksturi i boji}
\author{Iva Miholić, Kristijan Franković, Dragan Drandić, Gustav Matula, Ivan Katanić}

\begin{document}
\maketitle

\chapter{Detekcija pješaka u urbanim okruženjima}
\section{Opis zadatka}
Detekcija pješaka u urbanim okruženjima problem je povezan sa automobilskom industrijom, sigurnošću u prometu, sigurnosnim nadziranjem i robotikom. Rješava se kao detekcija objekta u okviru područja računalnog vida.

Pregled najznačajnijih rješenja ovog problema dan je u \cite{BenensonOHS14}. U ovom će radu biti riječi o detektorima temeljenih na pretraživanju graničnih prozora.  
%TODO opis poznatih metoda

\chapter{Baza podataka za treniranje i verifikaciju rješenja}

Za treniranje i verifikaciju rješenja koristit ćemo skup podataka INRIA \cite{DT05}. To je skup fotografija urbanog okruženja u boji i pripadnih anotacija. Za svaku fotografiju zabilježen je skup graničnih prozora \engl{bounding window} unutar kojih se nalaze prikazi uspravnih osoba. %TODO primjer oznacene slike i kako se definira bounding window

INRIA se do sada često koristio za trening detektora zbog raznolikosti pozadinskih okruženja osoba na slikama i točnosti anotacija \cite{BenensonOHS14}. Za razliku od ostalih javno dostupnih baza podataka za detekciju pješaka, ove fotografije nisu dobivene iz videa te su relativno visoke kvalitete i fotografirane su iz različitih točaka gledišta. U drugim bazama podataka, pješaci su većinom konecentrirani u centralnoj horizontali fotografije jer je ista dobivena iz vozačeve perspektive.

Baza se sastoji od podskupa za trening i podskupa za evaluaciju. U podskupu za trening označena su $1208$ pješaka u $614$ od $1832$ fotografije. U podskupu za testiranje označeno je $566$ pješaka u $453$ od $741$ fotografije \cite{Dollar:2012:PDE:2197081.2197275}

\chapter{Idejno rješenje i prikaz arhitekture sustava računalnog vida}

\section{Pregled značajki temeljenih na teksturi i boji}

\section{Plan arhitekture sustava računalnog vida}
\bibliographystyle{plain}
\bibliography{bibliografija}
\end{document}